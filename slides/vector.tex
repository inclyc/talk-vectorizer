\documentclass[aspectratio=169]{ctexbeamer}
% \setbeamertemplate{bibliography item}{\insertbiblabel}
\usepackage{booktabs}
\usetheme{Madrid}

\title{Auto-Vectorization \& SIMD in LLVM}
\subtitle{High Performance Computing}

\input{section_pages.tex}
\input{tuna_color.tex}

\author{Y.C. Long}



\begin{document}

\begin{frame}
    \maketitle
\end{frame}

\begin{frame}
    \frametitle{Table of Contents}

    \tableofcontents

\end{frame}

\section{Moore定律的终结和并行化的需求}

\begin{frame}
    \frametitle{Moore's Law to be over}

    集成电路上可容纳的晶体管数目,约每隔两年便会增加一倍

    \begin{figure}[h]
        \includegraphics[height=0.5\textheight]{images/moore.jpeg}
        \caption{Moore's Law}
    \end{figure}

\end{frame}

\begin{frame}
    \frametitle{Parallel}

    提高计算机性能的方法,是并行化。

    \begin{figure}[h]
        \includegraphics[height=0.5\textheight]{images/dual_core.png}
        \caption{Dual Core Cache Design}
    \end{figure}

\end{frame}

\section{并行的手段}

\subsection{Multi-Core}

\begin{frame}
    \frametitle{Multi-Core}

    为了实现并行化,我们可以给一个计算机加入多个核心。

    \begin{itemize}
        \item 不同的寄存器
        \item 不同的中断处理请求
        \item 操作系统-对称多处理(SMP)调度
    \end{itemize}

    \begin{figure}[h]
        \includegraphics[height=0.45\textheight]{images/smp.png}
        \caption{Symmetric multiprocessing}
    \end{figure}

\end{frame}

\subsection{Single-Core}

\begin{frame}
    \frametitle{Single-Core - Out-of-Order Execution}

    乱序执行(Out-of-Order Execution)是现代CPU最基本的一个并行手段。

    % TODO: 展示乱序执行的例子

\end{frame}

\begin{frame}
    \frametitle{Single-Core - SIMD}

    OoOE在编程上由编译器全局指令调度器(Instruction Scheduler)优化。

    单指令流多数据流(Single instruction, multiple data (SIMD)),提供了一种让我
    们更好地进行向量计算的方式。

    % TODO: SIMD graph

\end{frame}

\begin{frame}
    \frametitle{SIMD - Videos \& Games}

    % TODO: SIMD 在游戏、视频中的应用

\end{frame}

\end{document}
